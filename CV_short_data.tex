%% start of file `template_en.tex'.
%% Copyright 2007 Xavier Danaux (xdanaux@gmail.com).
%
% This work may be distributed and/or modified under the
% conditions of the LaTeX Project Public License version 1.3c,
% available at http://www.latex-project.org/lppl/.


\documentclass[10pt,a4paper]{moderncv}

\moderncvtheme[green]{banking}           
% style options are 'casual' (default), 'classic', 'oldstyle' and 'banking'

\usepackage[utf8]{inputenc}                   
\usepackage[scale=0.9]{geometry}
\AtBeginDocument{\recomputelengths}              

\usepackage{multicol}

\def\blue{\textcolor{gray!30!blue!40!}}
\def\etal{\emph{et al.~}}

\renewcommand*{\namefont}{\fontsize{25}{40}\mdseries\upshape}
\renewcommand*{\titlefont}{\fontsize{12}{40}\mdseries\upshape}

  
\firstname{Jos\'e-Alberto}
\familyname{V\'azquez}

\title{Brookhaven National Lab, U.S. Department of Energy}               
\address{Brookhaven Nat. Lab.}{Bdlg. 510-A, Upton NY. 11973} 
\phone{+1 631 344 4060} {}                   
%\fax{1223 337523}                          
\email{jv292@cam.ac.uk}                    
\email{jvazquez@bnl.gov}
\mobile{+1 631 992 0730}                    
%\extrainfo{JA Vazquez} 
%\photo[74pt]{yo}                         % '64pt'
\quote{Some quote (optional)}                
%\nopagenumbers{}                             
                         


%----------------------------------------------------------------------------------
%            content
%----------------------------------------------------------------------------------

\begin{document}
\maketitle
%
%------------------------------------------------------------------------------------------------------------
\vspace{-0.7cm}
\section{\hspace{0.5cm}Current Position}
\cventry{Oct.2013 - }{Post-doctoral Research Associate. Prof.~A. Slosar.}{Brookhaven National Lab, U.S. Department of Energy.}{NY, US}{}{}{}
%\cvline{}{\emph{``Cosmological Implications of BAO measurements and Lyman-$\alpha$ forest analysis".}}{}{}{}


\vspace{0.3em}
\cvline{}{I started my career as a theoretical cosmologist and slowly moving to statistics, astrophysics and data analysis. 
I am a post-doctoral research associate at the Brookhaven National Lab (BNL), and member of the Sloan Digital Sky Survey (SDSS).}

\cvline{}{My current interests are mainly focussed on the construction of a map of the universe 
based on the statistical properties of the Galaxies observed through the SDSS survey; 
on the parameter estimation and model selection of cosmological models through Bayesian statistical techniques.} 

\vspace{0.3em}
\subsection{ Interests}
\cvline{}{Programming, Data analysis, MCMC, Bayesian Statistics, Cosmology.
\hfill \href{http://www.cosmo.bnl.gov/www/jvazquez/}{\blue{Website}}, 
		\href{https://www.linkedin.com/in/jalbertovazquez}{\blue{LinkedIn}}, 
		\href{https://github.com/ja-vazquez}{\blue{GitHub}} }
		
%---------------------------------------------------------------------------------------------------------------------------------------------------------

\vspace{0.3em}
\section{\hspace{0.5cm}Education and Work Experience}

\vspace{0.3em}
\cventry{2009-2013}{Ph.D.  in Astrophysics, Prof. A. Lasenby and Prof. M. Hobson.}{KICC, University of Cambridge.}{Cambridge, UK}{}{}{}
\cvline{}{\emph{``Constraining alternative cosmological models with current and future observations''.}}
\cvline{}{We present Bayesian  techniques -Bayesian evidence, nested sampling and neural network algorithms-  
to  compare cosmological models in the light of the currently available data and forecasts for the next generation of experiments.}


\vspace{0.3em}
\cventry{2008-2009}{MASt. in Mathematics, Dr. A. Challinor.}{DAMTP, University of Cambridge.}{Cambridge, UK}{}{}
%\cvline{}{\emph{``Constraining cosmological Inflation''.} }
%\cvline{}{ With the use of current
%and future surveys, we show constraints on the Inflationary parameters that allow us to make 
%the connection between theoretical and observational cosmology.}


\vspace{0.3em}
\cventry{2005-2007}{M.Sc. in Physics, Prof. T. Matos.}{Physics Department, CINVESTAV.}{DF, MX}{}{}{}
%\cvline{}{\emph{``Dynamical systems in Scalar Field Cosmologies''.}} 
%\cvline{}{ We use the dynamical systems formalism to study the stability of scalar fields as the main candidates of Dark Matter.}

\vspace{0.3em}
\cventry{2000-2005}{B.Sc. in Physics, Prof. T. Matos.}{Faculty of Sciences, UAEMor.}{Morelos, MX}{}{}{}
%\cvline{}{\emph{``Galaxy formation with scalar-field dark matter''.}}


\vspace{0.5em}
\subsection{Research Internships}

\vspace{0.3em}
\cventry{Jun-Oct.13'}{Visiting Researcher, hosted by Prof. T. Matos.}{Physics Department, CINVESTAV.}{ DF, MX}{}{}{}
%\cvline{}{Lecture on `General cosmology', 
%and mentor three master students in their summer projects.}

\vspace{0.3em}
\cventry{2007-2008}{Graduate Research Assistant, hosted by Prof. T. Matos.}{Physics Department, CINVESTAV.}{DF, MX}{}{}{}
%\cvline{}{\emph{``Cosmological models with dynamical systems''.}}

\vspace{0.3em}
\cventry{Jun-Sept.06'}{Short-term research visitor, hosted by Prof. B. Brugmann.}{Friedrich-Schiller-Universitat Jena.}{Jena, DE}{}{}{}{}
%\cvline{}{\emph{``Numerical methods in Cosmology".} }

%----------------------------------------------------------------------------------------------------------------------------------------------------


\vspace{0.3em}
\section{\hspace{0.5cm}Selected Awards \& Scholarships  }
\cvline{2015}{Invited for a plenary talk on behalf of the BOSS collaboration to the APS meeting.}
\cvline{2014}{Invited for a plenary talk on behalf of the BOSS collaboration to the SDSS-IV meeting.}
 \cvline{2013}{National System of Researchers, Level 1 (SNI 1).}
\cvline{2013}{PhD award for academic purposes, Cavendish Laboratory, Cambridge.}
%($\textsterling$ 3k)}
\cvline{2012}{Tutorial award for academic purposes, St Edmund's College, Cambridge.}
%($\textsterling$ 2k)}{}{}
\cvline{2012}{American Alumni award, for traveling to the US for studies. St Edmund's College, Cambridge.}
%($\textsterling$ 1k)}
\cvline{2008-2012}{SEP Excellence program scholarship, complementary scholarship.}
%($\sim \$$15k)}{}{}
\cvline{2008-2012}{CONACyT full scholarship, for  study towards a MASt and PhD, U. of Cambridge.}
%($\sim \textsterling$  90k)}{}{}
\cvline{2006}{Research grant for young scientists. \emph{Awarded by the German Academic Exchange Service (DAAD).} }
\cvline{2005-2007 }{CONACyT full scholarship, for  study towards a Master, CINVESTAV.}
%($\sim$ MX 195k)}{}{}
\cvline{2004-2005}{Undergraduate Research Assistantship (from SNI-III), UAEM-CINVESTAV.}{}{}{}
 \cvline{2004 }{Undergraduate Teaching Assistantship,  UAEM.}
 
 \vspace{0.3em}
 \subsection{Press Release}
\cvline{07.2016}{US Department of Energy: Dark Energy Measured With Record-Breaking Map of 1.2 Million Galaxies
	\hfill \href{https://www.bnl.gov/newsroom/news.php?a=11854}{\blue{link}}}
\cvline{07.2016}{LBNL, Berkeley Lab: Dark Energy Measured with Record-Breaking Map of 1.2 Million Galaxies
	\hfill \href{http://newscenter.lbl.gov/2016/07/14/record-breaking-map-1-2-million-galaxies/}{\blue{link}.}}
\cvline{07.2016}{Physicsworld: Dark-energy study maps 1.2 million galaxies in the early universe
	\hfill \href{http://physicsworld.com/cws/article/news/2016/jul/15/dark-energy-study-maps-1-2-million-galaxies-in-the-early-universe}{\blue{link}.}}
\cvline{04.2015}{APS meeting on behalf of the BOSS Collaboration}
\cvline{06.2012}{Talented Mexicans abroad. TV. short interview (Televisa)}
 
  
 %----------------------------------------------------------------------------------------------------------------------------------------------------------

\vspace{0.3em} 
 \section{\hspace{0.5cm} Academic experience}
 \cvline{Publications } {Author of 23 publications in distinguished 
 journals, three conference proceedings and two science review papers. 
 Over half of the papers as a principal author, and two of them leading a 
 collaboration of more than a hundred author-paper. 
 	%\href{http://www.cosmo.bnl.gov/www/jvazquez/files/Research.html}{ \blue{\underline{Research in progress.}}}} 
 %\cvline{}{
 \hfill
 For further details and citations: 
	\href{http://scholar.google.com/citations?user=PCuxBOkAAAAJ&hl=en}{\blue {Google Scholar}}, 
	\href{http://inspirehep.net/author/profile/J.A.Vazquez.1}{ \blue{Inspire}},
	\href{https://www.researchgate.net/profile/J_Vazquez3}{ \blue{Research gate}}}
\vspace{0.3em}

\vspace{0.3em}
\cvline{ Invited talks and travel grants}{I have been awarded with travel grants to attend conferences and workshops. 
Some of the institutions include: 
ICTP, Trieste, IT; Cambridge, UK; SLC, UT, USA; Cabo, MX; Harvard, MA, USA;  Passo del Tonale, IT; Stanford, CA, USA; AIE, Berlin, DE; Ensenada, MX;  CMU, PA; Aspen, CO;
ITU, Istanbul; Berkeley, CA; UNAM, MX.}
%\href{http://www.cosmo.bnl.gov/www/jvazquez/files/CV.html}{ \blue{\underline{Long-cv}}}

\vspace{0.3em}
 \cvline{Organization}{Workshop Organizer: ``Statistical and Numerical methods  in Cosmology" (50 participants), IF, UNAM.
Mini-workshop Organizer: ``Overview to CAMB and CosmoMC" (15 participants),  ININ. Seminar group Organiser:
``Geometry and Gravitation'', CINVESTAV. Seminar group Organizer: ``Cosmology, Astrophysics and Numerical relativity'', 
CINVESTAV.}


\vspace{0.3em}

\section{\hspace{0.5cm}Hacking}
\cvline{08.2016}{PyData.													\hfill Chicago, IL, US}
\cvline{07.2016}{PyGotham \& Database Camp.									\hfill UN. NY, US}
%\cvline{07.2016}{Database Camp.											\hfill NY, US}
\cvline{06.2016}{8th Astronomical Data Analysis Summer School.					\hfill Chania,  GR}
\cvline{01.2015}{Symposium and Hack Week on data-intensive cosmology.			\hfill  Berkeley, CA, US}
\cvline{04.2015}{SciCoder 6 Workshop.										\hfill  NYU, US}


%----------------------------------------------------------------------------------------------------------------------------------------------------
 \vspace{0.3em}

\section{\hspace{0.5cm}Skills and Interests}
\cvline{Programming Languages}{ \hfill Python, C/C++, Fortran, R, Bash Scripting}{}
\cvline{Maths}{\hfill Maple, Mathematica, Matlab (basic)}{}
\cvline{Op. Systems}{\hfill Linux, Windows, Mac OS X}
\cvline{Databases}{\hfill MySQL, SQLite}
\cvline{Useful}{\hfill Latex, HTML, CSS, Gnuplot, Git, SVN}

\vspace{0.3em}
\subsection{Packages, libraries and frameworks}
\cvline{Python}{\hfill Numpy, Pandas, Scipy, Scikit-learn, Beatiful Soup, Matplotlib, Bokeh, Seaborn, Flask.}
\cvline{R}{\hfill dplyr, Main ones for Stats and ML, ggplot2, Shiny}
\cvline{C/C++, Fortran}{\hfill  LAPACK, OpenMP, MPI}
\cvline{HPC Clusters}{\hfill NERSC(LBNL), Astro (BNL), Darwin (Cambridge), LaSuma-(CINVESTAV)}


\vspace{0.3em}

\subsection{Cosmology}
%\cvline{}{ CAMB, CosmoMC, MultiNest, CosmoNet, CosmoSIS, SimpleMC.}
\vspace{0.3em}
\cvline{Contributions}{ 
\hfill MCMC for BAO analysis for the BOSS collaboration (Python) - \hspace{1cm}  \href{https://github.com/ja-vazquez/SimpleMC}{\blue{SimpleMC}}\\. 
\hfill Massively parallelizable Gaussian Embedding Sampling (Python) - \hspace{0.4cm}  \href{https://github.com/ja-vazquez/GM_Sampler}{\blue{GM algorithm}}\\.
\hfill Model Independent Bayesian Reconstruction (Fortran) - \hspace{1.cm} \href{http://www.mrao.cam.ac.uk/facilities/software/np-camb/}{\blue{NP-CAMB}}}
\hfill Lyman-$\alpha$ analysis for the BOSS collaboration (C++) - \hspace{0.9cm} \href{http://c3.lbl.gov:8000/Trac.Cosmology/wiki/CosmologyCodeInstallation}{\blue{Cosmology}}

\vspace{0.3em}
%------------------------------------------------------------------------------------------------------------

\subsection{Non-Academic Projects}
\cvline{}{Scraping the web, Using APIs, Data manipulation with Pandas and SQL, 
	Playing with Stats and ML algorithms and Visualizations.
	 \hfill For further details see: 
	\href{https://github.com/ja-vazquez}{\blue {GitHub}},
	\href{https://bitbucket.org/ja_vazquez/}{\blue {Bitbucket}}.	
	}

\cvline{Meetups}{I regularly attend NYC meetups with keywords such as Python, R, SQL, Data science. }



\vspace{0.3em}

\subsection{Others}
\cvline{Languages}{\hfill  Spanish (Native);  English (Fluent);  German (Elementary). }{}
\cvline{ Sports}{\hfill Football (participation on national tournaments), Squash, Climbing, Jogging, Cycling.}
\cvline{}{\hfill Organiser of the national football tournament of Mexican Societies  in UK (05.2010).}
\cvline{Others}{ \hfill Reading: Economy,  Science, Science Fiction; Board games: Chess, Backgammon, Poker.}{}	


%\vspace{.em}
%\cvline{}{\hfill JAV}{}{}{}



 



\closesection{}                   % needed to renewcommands
\renewcommand{\listitemsymbol}{-} % change the symbol for lists
%------------------------------------------------------------------------------------------------------------	
	


%------------------------------------------------------------------------------------------------------------
%------------------------------------------------------------------------------------------------------------
%------------------------------------------------------------------------------------------------------------
%------------------------------------------------------------------------------------------------------------	
%\section{Current Research}
%{The research I am currently carrying out has two main  aims: The first is to generate
%predictions of  observable quantities, within cosmological models incorporating extra
%features beyond the standard Lambda Cold Dark Matter ($\Lambda$CDM). The predictions
%are sought for the  expansion history  and the generation and evolution of perturbations.
%Secondly, advanced  Bayesian  techniques, involving the calculation of evidence and 
%nested sampling,  are carried out to  compare cosmological models using  actual data, including
%the Cosmic Microwave Background, Large-Scale  Structure and Supernovae. Future  surveys
%like  Planck and CMBPol  are also  taking  part in the analysis.}
%\vspace{0.5em}
	
%------------------------------------------------------------------------------------------------------------		
	
%\section{Personal Information}
%\cvline{Name}{Jos\'e-Alberto V\'azquez-Gonz\'alez}
%\cvline{Date of birth}{September 6, 1982} 
%\cvline{Place }{Cuernavaca, Morelos, Mexico}
%\cvline{Interests}{ BAO, CMB, Ly-$\alpha$ forest; Dark Energy, Inflation; Data analysis.} 
%\vspace{0.5em}
\end{document}


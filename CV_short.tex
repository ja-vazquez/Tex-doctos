%% start of file `template_en.tex'.
%% Copyright 2007 Xavier Danaux (xdanaux@gmail.com).
%
% This work may be distributed and/or modified under the
% conditions of the LaTeX Project Public License version 1.3c,
% available at http://www.latex-project.org/lppl/.


\documentclass[10pt,a4paper]{moderncv}

\moderncvtheme[blue]{casual}           
% style options are 'casual' (default), 'classic', 'oldstyle' and 'banking'

\usepackage[utf8]{inputenc}                   
\usepackage[scale=0.86]{geometry}
\AtBeginDocument{\recomputelengths}              

\usepackage{multicol}

\def\blue{\textcolor{blue}}
\def\etal{\emph{et al.~}}

\renewcommand*{\namefont}{\fontsize{25}{40}\mdseries\upshape}
\renewcommand*{\titlefont}{\fontsize{12}{40}\mdseries\upshape}

  
\firstname{Jos\'e-Alberto}
\familyname{V\'azquez}
\title{Brookhaven National Lab, U.S. Department of Energy}               
\address{Brookhaven Nat. Lab.}{Bdlg. 510-A, Upton NY. 11973} 
\phone{+1 631 344 4060} {}                   
%\fax{1223 337523}                          
\email{jv292@cam.ac.uk}                    
\email{jvazquez@bnl.gov}
\mobile{+1 631 992 0730}                    
\extrainfo{JA Vazquez} 
%\photo[74pt]{yo}                         % '64pt'
%\quote{Some quote (optional)}                
%\nopagenumbers{}                             


%----------------------------------------------------------------------------------
%            content
%----------------------------------------------------------------------------------

\begin{document}
\maketitle
%
%------------------------------------------------------------------------------------------------------------
\vspace{-1.cm}
\section{Current Position}
\cventry{Oct.2013 - }{Post-doctoral Research Associate}{\hfill Brookhaven National Lab}{NY, USA}{}{}{}
\cvline{}{Member of the SDSS-III/SDSS-IV collaboration.}
\cvline{}{Project: \emph{``Cosmological Implications of BAO measurements and Lyman-$\alpha$ forest analysis"}}{}{}{}
\vspace{0.3em}

\cvline{}{My current research is mainly focussed on the data analysis of the Lyman-$\alpha$ forest observed through BOSS; 
on the parameter estimation and model selection of Dark Energy and Inflationary models.} 
\cvline{} {\href{http://www.cosmo.bnl.gov/www/jvazquez/}{\blue{\underline{Website}}}, 
		\href{https://www.linkedin.com/in/jalbertovazquez}{\blue{\underline{LinkedIn}}}, 
		\href{https://github.com/ja-vazquez}{\blue{\underline{GitHub}}} }
%---------------------------------------------------------------------------------------------------------------------------------------------------------

\section{Education and Work Experience}

\cventry{2009-2013}{Ph.D.  in Astrophysics}{\hfill KICC}{University of Cambridge, UK}{}{}{}
\cvline{}{\emph{``Constraining alternative cosmological models with current and future observations''.}
We present advanced  Bayesian  techniques -the computation of the Bayesian evidence, 
nested sampling and neural network algorithms-  to  compare cosmological models in the light of the currently 
available data and forecasts for the next generation of experiments.
}{}{}{}
\vspace{0.3em}

\cventry{2008-2009}{MASt. in Mathematics}{\hfill DAMTP, University of Cambridge, UK}{}{}{}
\cvline{}{\emph{``Constraining cosmological Inflation''.} With the use of current
and future surveys, we show constraints on the Inflationary parameters that allow us to make 
the connection between theoretical and observational cosmology.}
\vspace{0.3em}

\cventry{2005-2007}{M.Sc. in Physics}{\hfill Physics Department}{CINVESTAV, MX}{}{}{}
\cvline{}{\emph{``Dynamical systems in Scalar Field Cosmologies''.} We use
the dynamical systems formalism to study the stability of scalar fields as the main candidates of Dark Matter.}
\vspace{0.3em}

\cventry{2000-2005}{B.Sc. in Physics}{\hfill Faculty of Sciences}{UAEMor, MX}{}{}{}
\cvline{}{\emph{``Galaxy formation with scalar-field dark matter''.} We present the general
picture of the Dark matter and study several candidates, in particular single scalar fields.}
\vspace{0.0em}


\subsection{Research Internships}
\cventry{Jun-Oct.13'}{Visiting Researcher}{\hfill Physics Department, CINVESTAV}{ MX}{}{}{}
\cvline{}{Collaboration visit where I present several lectures about `General cosmology' in the physics department, 
and mentored three master students in their summer projects. }
\vspace{0.3em}

\cventry{2007-2008}{Graduate Research Assistant}{\hfill Physics Department}{CINVESTAV, MX}{}{}{}
\cvline{}{\emph{``Cosmological models with dynamical systems''.} Prof. T. Matos.}
\vspace{0.3em}

\cventry{Jun-Sept.06'}{Short-term research visitor}{\hfill Friedrich-Schiller-Universitat}{Jena, DE}{}{}{}{}
\cvline{}{\emph{``Numerical methods in Cosmology".} The aim of this visit was to learn and work with 
numerical techniques in order to solve basic gravitational wave equations.} 

%----------------------------------------------------------------------------------------------------------------------------------------------------
\vspace{0.0em}


\section{Selected Awards \& Scholarships  }
\cvline{2015}{Selected to present a plenary
talk on behalf of the BOSS collaboration to the American Physical Society (APS) meeting 2015.}
\cvline{2014}{Selected to present a plenary
talks on behalf of the BOSS collaboration to the SDSS-IV meeting.}
 \cvline{2013}{National System of Researchers, Level 1 (SNI 1).}
\cvline{2013}{PhD award for academic purposes, Cavendish Laboratory, Cambridge.}
%($\textsterling$ 3k)}
\cvline{2012}{Tutorial award for academic purposes, St Edmund's College, Cambridge.}
%($\textsterling$ 2k)}{}{}
\cvline{2012}{American Alumni award, for traveling to the US for studies. St Edmund's College, Cambridge.}
%($\textsterling$ 1k)}
\cvline{2008-2012}{SEP Excellence program scholarship, complementary scholarship.}
%($\sim \$$15k)}{}{}
\cvline{2008-2012}{CONACyT full scholarship, for  study towards a MASt and PhD, University of Cambridge.}
%($\sim \textsterling$  90k)}{}{}
\cvline{2006}{Research grant for young scientists. \emph{Awarded by the German Academic Exchange Service (DAAD).} }
\cvline{2005-2007 }{CONACyT full scholarship, for  study towards a Master, CINVESTAV.}
%($\sim$ MX 195k)}{}{}
\cvline{2004-2005}{Undergraduate Research Assistantship (from SNI-III), UAEM-CINVESTAV.}{}{}{}
 \cvline{2004 }{Undergraduate Teaching Assistantship,  UAEM.}
 
 
 %----------------------------------------------------------------------------------------------------------------------------------------------------------
%\vspace{0.0em}


% \section{Affiliations}
% \cvline{2015 --}{Member of the APS, AAS.}
%  \cvline{2014 --}{Member of the Advisory Committee for CONACYT Mexico (RCEA).
%  Referee of projects: `Installation of a high energy and astroparticle lab', asking for \$US 300k; and 
%  `Physics and astrophysics of neutron stars', asking for \$US 200k.}
% \cvline{2013 --}{Member of the SDSS-III/SDSS-IV collaboration, as part of the BOSS/eBOSS experiment.}
%   \cvline{2012 --}{Committee member of the Mexican Cambridge Society - web master and sports officer-.}
% \cvline{2006 --}{Member of the Institute advanced  of  cosmology, \url{http://www.iac.edu.mx/}}{}{}{}
% \cvline{2004-2005}{Counselor student at Graduate Internal Council, UAEM.}{}{}{}
  
 %----------------------------------------------------------------------------------------------------------------------------------------------------------
\vspace{0.0em}
 
 \section{Publications \& Academic experience}
 \cvline{Publications } {Author of 20 publications in distinguished 
 journals, two conference proceedings and two science review papers. 
 Over half of the papers as a principal author, and one of them leading a 
 collaboration of more than a hundred author-paper. 
 	\href{http://www.cosmo.bnl.gov/www/jvazquez/files/Research.html}{ \blue{\underline{Research in progress.}}}} 
 \cvline{*}{For further details and citations: 
	\href{http://scholar.google.com/citations?user=PCuxBOkAAAAJ&hl=en}{\blue {\underline{Google Scholar}}}, 
	\href{http://inspirehep.net/author/profile/J.A.Vazquez.1}{ \blue{\underline{Inspire}}},
	\href{https://www.researchgate.net/profile/J_Vazquez3}{ \blue{\underline{Research gate}}}}
\vspace{0.3em}


 \cvline{Invited talks }{
 I have presented my research throughout several talks, but in particular I was invited to give plenary
talks on behalf of the BOSS collaboration to the American Physical Society (APS) meeting 2015 and to the SDSS-III
and SDSS-IV 2014 collaboration meetings.
 Other institutions include: 
 CMU, PA; Aspen, CO;
ITU, Istanbul; Berkeley, CA; UNAM, MX; CINVESTAV, MX; Cambridge, UK.}
\vspace{0.3em}

\cvline{ Travel grants}{I have also been awarded with travel grants to attend conferences and workshops, and
present shorts a short talk. Some of the institutions include: 
ICTP, Trieste, IT; Cambridge, UK; SLC, UT, USA; Cabo, MX; Harvard, MA, USA;  Passo del Tonale, IT; Stanford, CA, USA; AIE, Berlin, DE; Ensenada, MX.}
\vspace{0.3em}

 \cvline{Organization}{Workshop Organiser: ``Statistical and Numerical methods  in Cosmology" (50 participants), IF, UNAM.
Mini-workshop Organiser: ``Overview to CAMB and CosmoMC" (15 participants),  ININ. Seminar group Organiser:
``Geometry and Gravitation'', CINVESTAV. Seminar group Organiser: ``Cosmology, Astrophysics and Numerical relativity'', 
CINVESTAV.}
 
 \cvline{Hacking}{(01.2015) Symposium and Hack Week on data-intensive cosmology.  Berkeley, CA, USA.
	\href{http://bccp.berkeley.edu/workshops/ctu-2015/}{\blue {\underline{link}}}}
\cvline{}{(04.2015) SciCoder 6 Workshop.	  NY, USA.
	\href{http://www.scicoder.org/Welcome.html}{\blue {\underline{link}}}}
\cvline{}{Statistical methods for cosmology, Astrostatistics and R (Eric Feigelson, PennState U.)}
\cvline{}{Applied Bayesian Statistics - with R (David Spiegelhalter, Cambridge)}
\cvline{}{Bayesian methods in Cosmology (Mike Hobson -PhD advisor - Cambridge) }
\vspace{0.3em}
%----------------------------------------------------------------------------------------------------------------------------------------------------
 \vspace{0.0em}


\section{Skills and Interests}
\cvline{}{I have been involved in several projects where programming skills are a key factor. Some of the programming
languages and software that I used the most include:}
\cvcomputer{Programming}{ C/C++, Fortran,  Python, R (basic)}{}{Matlab, Maple (basic), Mathematica.}
\cvcomputer{Useful}{OpenMP, MPI, mpi4py}{}{Git/svn, Markdown, Markup(Latex/HTML)}
\cvline{Op. Systems}{Unix, Linux, Windows, Mac OS X.}

\vspace{0.3em}
\cvline{Cosmo}{ CAMB, CosmoMC, MultiNest, CosmoNet, CosmoSIS, SimpleMC.}
\cvline{Contributions}{ \href{https://github.com/ja-vazquez/SimpleMC}{\blue{\underline{SimpleMC}}}(MCMC for BAO analysis in BOSS),
	\href{https://github.com/ja-vazquez/GM_Sampler}{\blue{\underline{GE algorithm}}} (New Sampling Algorithm)}
\cvline{}{ \href{http://c3.lbl.gov:8000/Trac.Cosmology/wiki/CosmologyCodeInstallation}
	{\blue{\underline{Cosmology}}} (BOSS Lyman-$\alpha$ analysis),
\href{http://www.mrao.cam.ac.uk/facilities/software/np-camb/}{\blue{\underline{NP-CAMB}}} (Bayesian Reconstruction).}

 \cvline{*}{For further details see: 
	\href{https://github.com/ja-vazquez}{\blue {\underline{GitHub}}},
	\href{https://bitbucket.org/ja_vazquez/}{\blue {\underline{Bitbucket}}}
	}

\vspace{0.0em}
\subsection{Others}
\cvline{Languages}{Native Spanish; fluent English; elementary German. }{}
\cvline{ Sports}{Football (participation on national tournaments), Squash, Climbing, Jogging, Cycling.}
\cvline{}{Organiser of the national football tournament of Mexican Societies  in UK -150 attendees- (05.2010).}
\cvline{Others}{Reading: Economy,  Science, Science Fiction; Board games: Chess, Backgammon, Poker.}{}	
\cvline{Extra}{Online trader (Stocks/ETFs).}


 



\closesection{}                   % needed to renewcommands
\renewcommand{\listitemsymbol}{-} % change the symbol for lists
%------------------------------------------------------------------------------------------------------------	
	


%------------------------------------------------------------------------------------------------------------
%------------------------------------------------------------------------------------------------------------
%------------------------------------------------------------------------------------------------------------
%------------------------------------------------------------------------------------------------------------	
%\section{Current Research}
%{The research I am currently carrying out has two main  aims: The first is to generate
%predictions of  observable quantities, within cosmological models incorporating extra
%features beyond the standard Lambda Cold Dark Matter ($\Lambda$CDM). The predictions
%are sought for the  expansion history  and the generation and evolution of perturbations.
%Secondly, advanced  Bayesian  techniques, involving the calculation of evidence and 
%nested sampling,  are carried out to  compare cosmological models using  actual data, including
%the Cosmic Microwave Background, Large-Scale  Structure and Supernovae. Future  surveys
%like  Planck and CMBPol  are also  taking  part in the analysis.}
%\vspace{0.5em}
	
%------------------------------------------------------------------------------------------------------------		
	
%\section{Personal Information}
%\cvline{Name}{Jos\'e-Alberto V\'azquez-Gonz\'alez}
%\cvline{Date of birth}{September 6, 1982} 
%\cvline{Place }{Cuernavaca, Morelos, Mexico}
%\cvline{Interests}{ BAO, CMB, Ly-$\alpha$ forest; Dark Energy, Inflation; Data analysis.} 
%\vspace{0.5em}
\end{document}


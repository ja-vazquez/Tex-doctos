%% start of file `template_en.tex'.
%% Copyright 2007 Xavier Danaux (xdanaux@gmail.com).
%
% This work may be distributed and/or modified under the
% conditions of the LaTeX Project Public License version 1.3c,
% available at http://www.latex-project.org/lppl/.


\documentclass[11pt,a4paper]{moderncv}

\moderncvtheme[green]{banking}           
% style options are 'casual' (default), 'classic', 'oldstyle' and 'banking'

\usepackage[utf8]{inputenc}                   
\usepackage[scale=0.8]{geometry}
\AtBeginDocument{\recomputelengths}              

\usepackage{multicol}

\def\blue{\textcolor{gray!30!blue!40!}}
\def\etal{\emph{et al.~}}

\renewcommand*{\namefont}{\fontsize{25}{40}\mdseries\upshape}
\renewcommand*{\titlefont}{\fontsize{12}{40}\mdseries\upshape}

  
\firstname{Jos\'e-Alberto}
\familyname{V\'azquez}

\title{Centro de Investigaci\'on y de Estudios Avanzados del IPN }               
\address{Av. IPN 2508, Gustavo A. Madero,}{San Pedro Zacatenco, 07360, CDMX} 
\phone{+1 631 344 4060} {}                   
%\fax{1223 337523}                          
\email{jv292@cam.ac.uk}                    
\email{jvazquez@bnl.gov}
\mobile{+1 631 992 0730}                    
%\extrainfo{JA Vazquez} 
%\photo[74pt]{yo}                         % '64pt'
%\quote{Some quote (optional)}                
%\nopagenumbers{}                             


%----------------------------------------------------------------------------------
%            content
%----------------------------------------------------------------------------------

\begin{document}
\maketitle
%
%------------------------------------------------------------------------------------------------------------
\vspace{-0.3cm}
\section{Current Position}
\cventry{Oct.2016 - }{Catedr\'atico CONACYT, Prof.~ O. Miranda \& Prof.~T. Matos.}{Centro de Investigaci\'on y de Estudios
Avanzados del IPN.}{CDMX, MX}{}{}{}
%\cvline{}{\emph{``Cosmological Implications of BAO measurements and Lyman-$\alpha$ forest analysis".}}{}#{}{}
\cvline{}{Member of the SDSS-III/SDSS-IV collaboration.}
\vspace{0.3em}
%An${ \breve{z}}$e
%{The application of the BAO technique to large cosmological surveys has enabled the first 
%percent-level measurements of absolute distances beyond the Milky Way. In combination with 
%CMB and SN data, these measurements yield impressively tight constraints on the 
%cosmic expansion history and correspondingly stringent tests of dark energy theories. 
%Over the next year, the strength of these tests will advance significantly with the final 
%results from BOSS and the CMB polarization and improved temperature maps from Planck. 
%In the longer term, BAO measurements will gain in precision and redshift range through 
%a multitude of ongoing or planned spectroscopic surveys, including SDSS-IV eBOSS, 
%HETDEX, DESI, WEAVE, Euclid, and WFIRST.}\\

\subsection{Research Interests}
\cvline{}{BAO, CMB, Ly-$\alpha$ forest;  Dark Energy, Inflation;  Data analysis.
\hfill \href{http://www.cosmo.bnl.gov/www/jvazquez/}{\blue{Website}}, 
		\href{https://www.linkedin.com/in/jalbertovazquez}{\blue{LinkedIn}}, 
		\href{https://github.com/ja-vazquez}{\blue{GitHub}}\\ .\hfill http://www.cosmo.bnl.gov/www/jvazquez/
	 }
	 
\subsection{Place and Birth date}
\cvline{}{ Cuernavaca, Morelos M\'exico. 06-Septiembre-1982.}
%---------------------------------------------------------------------------------------------------------------------------------------------------------
\vspace{0.5em}



\section{Education and Work Experience}
\cventry{2013-2016}{Post-doctoral Research Associate, Prof.~A. Slosar.}{Brookhaven National Lab, U.S. Department of Energy.}{NY, USA}{}{}
\cvline{}{\emph{``Cosmological Implications of BAO measurements and Lyman-$\alpha$ forest analysis"}}
\vspace{0.5em}

\cventry{2009-2013}{Ph.D.  in Astrophysics, Prof. A. Lasenby and Prof. M. Hobson.}{KICC, University of Cambridge.}{Cambridge, UK}{}{}{}
\cvline{}{\emph{``Constraining alternative cosmological models with current and future observations''.}}
%Here we present advanced  Bayesian  techniques, the computation of the Bayesian evidence, 
%nested sampling and neural network algorithms,  to  compare alternative cosmological models in the light of the currently 
%available data and forecasts for the next generation of experiments. 
%\href{http://www.cosmo.bnl.gov/www/jvazquez/files/Publications.html}{\blue{\underline{Thesis.}}}}{}{}{}
\vspace{0.5em}

\cventry{2008-2009}{MASt. in Mathematics, Dr. A. Challinor.}{DAMTP, University of Cambridge.}{Cambridge, UK}{}{}
\cvline{}{\emph{``Constraining cosmological Inflation''.} }
%With the use of current and future surveys, we show constraints on the Inflationary parameters that allow us to make 
%the connection between theoretical and observational cosmology.}
\vspace{0.5em}

\cventry{2005-2007}{M.Sc. in Physics, Prof. T. Matos.}{Physics Department, CINVESTAV.}{DF, MX}{}{}{}
\cvline{}{\emph{``Dynamical systems in Scalar Field Cosmologies''.}} 
%In this work, by using
%the dynamical systems formalism, we study the stability of scalar fields as the main candidates of Dark Matter.}
\vspace{0.5em}

\cventry{2000-2005}{B.Sc. in Physics, Prof. T. Matos.}{Faculty of Sciences, UAEMor.}{Morelos, MX}{}{}{}
\cvline{}{\emph{``Galaxy formation with scalar-field dark matter''.}}
% We present the general
%picture of the Dark matter and study several candidates, in particular single scalar fields.}
\vspace{0.5em}


\subsection{Research Internships}
\cventry{Jun-Oct.13'}{Visiting Researcher, hosted by Prof. T. Matos.}{Physics Department, CINVESTAV.}{ DF, MX}{}{}{}
\cvline{}{Collaboration visit to give a lecture on `General cosmology', 
and mentor three master students in their summer projects.}
\vspace{0.5em}

\cventry{2007-2008}{Graduate Research Assistant, hosted by Prof. T. Matos.}{Physics Department, CINVESTAV.}{DF, MX}{}{}{}
\cvline{}{\emph{``Cosmological models with dynamical systems''.}}
\vspace{0.5em}

\cventry{Jun-Sept.06'}{Short-term research visitor, hosted by Prof. B. Brugmann.}{Friedrich-Schiller-Universitat Jena.}{Jena, DE}{}{}{}{}
\cvline{}{\emph{``Numerical methods in Cosmology".} }
%The aim of this visit was to learn and work with 
%numerical techniques in order to solve basic gravitational wave equations. } 

%----------------------------------------------------------------------------------------------------------------------------------------------------
\vspace{0.0em}


\section{Selected Awards \& Scholarships}
\cvline{2015}{Invited for a plenary talk on behalf of the BOSS collaboration to the APS meeting.}
\cvline{2014}{Invited for a plenary talk on behalf of the BOSS collaboration to the SDSS-IV meeting.}
\cvline{2013}{Member of the National System of Researchers, Level 1 (SNI 1).}
\cvline{2013}{PhD award for academic purposes, Cavendish Laboratory, Cambridge.}
%($\textsterling$ 3k)}
\cvline{2012}{Tutorial award for academic purposes, St Edmund's College, Cambridge.}
%($\textsterling$ 2k)}{}{}
\cvline{2012}{American Alumni award, for traveling to the US for studies. St Edmund's College, Cambridge.}
%($\textsterling$ 1k)}
\cvline{2008-2012}{SEP Excellence program scholarship, complementary scholarship.}
%($\sim \$$15k)}{}{}
\cvline{2008-2012}{CONACyT full scholarship, for  study towards a MASt and PhD, University of Cambridge.}
%($\sim \textsterling$  90k)}{}{}
\cvline{2006}{Research grant for young scientists. \emph{Awarded by the German Academic Exchange Service (DAAD).} }
\cvline{2005-2007 }{CONACyT full scholarship, for  study towards a Master, CINVESTAV.}
%($\sim$ MX 195k)}{}{}
\cvline{2004-2005}{Undergraduate Research Assistantship (from SNI-III tutor), UAEM-CINVESTAV.}{}{}{}
   \cvline{2004 }{Undergraduate Teaching Assistantship,  UAEM.}
%----------------------------------------------------------------------------------------------------------------------------------------------------------

\subsection{Press Release}
\cvline{07.2016}{US Department of Energy: Dark Energy Measured With Record-Breaking Map of 1.2 Million Galaxies.
	\hfill \href{https://www.bnl.gov/newsroom/news.php?a=11854}{\blue{link}}}
\cvline{07.2016}{LBNL, Berkeley Lab: Dark Energy Measured with Record-Breaking Map of 1.2 Million Galaxies.
	\hfill \href{http://newscenter.lbl.gov/2016/07/14/record-breaking-map-1-2-million-galaxies/}{\blue{link}.}}
\cvline{07.2016}{Physicsworld: Dark-energy study maps 1.2 million galaxies in the early universe.
	\hfill \href{http://physicsworld.com/cws/article/news/2016/jul/15/dark-energy-study-maps-1-2-million-galaxies-in-the-early-universe}{\blue{link}.}}
\cvline{04.2015}{APS meeting on behalf of the BOSS Collaboration.
	\hfill \href{http://meetings.aps.org/Meeting/APR15/Session/Z2?showAbstract}{\blue{link}}}
\cvline{06.2012}{Talented Mexicans abroad. TV. short interview (Televisa).}


\vspace{0.5em}

 \section{Affiliations}
 \cvline{2015 --}{Member of the APS, AAS.}
  \cvline{2014 --}{Member of the Advisory Committee for CONACYT projects (RCEA), by invitation.}
  \cvline{}{ \small{Referee of projects: `Installation of a high energy and astroparticle lab', asking for \$US 300k; and 
  `Physics and astrophysics of neutron stars', asking for \$US 200k.}}
 \cvline{2013 --}{Member of the SDSS-III/SDSS-IV collaboration, as part of the BOSS/eBOSS experiment.}
   \cvline{2012 --}{Committee member of the Mexican Cambridge Society.}
 \cvline{2006 --}{Member of the Institute advanced  of  cosmology, \url{http://www.iac.edu.mx/}}{}{}{}
  \cvline{2004 --2005}{Counselor student at Graduate Internal Council, UAEM.}{}{}{}
%----------------------------------------------------------------------------------------------------------------------------------------------------


 \vspace{0.5em}

\section{Publications}
\cvline{}{For further details and citations: 
	\href{http://scholar.google.com/citations?user=PCuxBOkAAAAJ&hl=en}{\blue {Google Scholar} (Cites:816)}, 
	\href{http://inspirehep.net/author/profile/J.A.Vazquez.1}{ \blue{Inspire} (Cites: 577)},
	\href{https://www.researchgate.net/profile/J_Vazquez3}{ \blue{Research gate}}}


 \vspace{0.3em}
\cvline{[1]  Galaxy-galaxy lensing estimators and their covariance properties:}
	{Sukhdeep Singh, Rachel Mandelbaum, Uro${ \breve{s}}$ Seljak, An${ \breve{z}}$e Slosar, JAV. 
	\hfill \href{https://arxiv.org/abs/1611.00752} {\blue{ArXiv:1611.00752}}}

 \vspace{0.3em}
\cvline{[2]  The Thirteenth Data Release of the Sloan Digital Sky Survey: 
First Spectroscopic Data from the SDSS-IV Survey MApping Nearby Galaxies at Apache Point Observatory}
	{Franco D. Albareti \etal Cites: 1. \hfill \href{https://arxiv.org/abs/1608.02013}{\blue{ArXiv:1608.02013}}}

 \vspace{0.3em}
\cvline{[3]  The clustering of galaxies in the completed SDSS-III Baryon Oscillation Spectroscopic Survey: 
		cosmological analysis of the DR12 galaxy sample}
	{Shadab Alam \etal Cites: 48. \hfill \href{https://arxiv.org/abs/1607.03155}{\blue{ ArXiv:1607.03155}}}

\vspace{0.3em}
\cvline{[4]  The clustering of galaxies in the completed SDSS-III Baryon Oscillation Spectroscopic Survey: 
	double-probe measurements from BOSS 	galaxy clustering \& Planck data -- towards an analysis without informative priors }
	{Marcos Pellejero-Ibanez \etal \hfill \href{https://arxiv.org/abs/1607.03152}{\blue{ ArXiv:1607.03152}} }

\vspace{0.3em}
\cvline{[5] The Clustering of Galaxies in the Completed SDSS-III Baryon Oscillation Spectroscopic Survey: 
	single-probe measurements from DR12 galaxy clustering -- towards an accurate model}
	{Chia-Hsun Chuang \etal Cites: 2. \hfill \href{https://arxiv.org/abs/1607.03151}{\blue{ArXiv:1607.03151}} }

\vspace{0.3em}
\cvline{[6] The clustering of galaxies in the completed SDSS-III Baryon Oscillation Spectroscopic Survey: 
		Baryon Acoustic Oscillations in Fourier-space}
	{Florian Beutler \etal Cites: 13. \hfill \href{https://arxiv.org/abs/1607.03149}{ArXiv:1607.03149.
	 \blue{MNRAS 464 (3): 3409-3430.} }}

\vspace{0.3em}
\cvline{[7] The clustering of galaxies in the completed SDSS-III Baryon Oscillation Spectroscopic Survey: 
	combining correlated Gaussian posterior distributions}
	{\\Ariel G. Sanchez \etal Cites: 4. 
	\hfill \href{https://arxiv.org/abs/1607.03146}{ArXiv:1607.03146. \blue{ MNRAS 464 (2): 1493-1501. }} }

\vspace{0.3em}
\cvline{[8] Constraining the dark energy equation of state using Bayes theorem and the Kullback-Leibler divergence}
	{S. Hee \etal \hfill \href{https://arxiv.org/abs/1607.00270}{\blue{ArXiv:1607.00270}}}

\vspace{0.3em}
\cvline{[9] Hybrid Natural Inflation}
	{\\Graham G. Ross, Gabriel German, JAV. Cites: 4. \hfill \href{https://arxiv.org/abs/1601.03221}{ArXiv:1601.03221.} 
	\blue{JHEP 1605 (2016) 010} }

 \vspace{0.3em}
\cvline{[10]  Broadband distortion modeling in Lyman-$\alpha$ forest BAO fitting}{ \\ Michael Blomqvist \etal
			Cites: 3. 
		   \hfill \href{http://arxiv.org/abs/1504.06656}{ArXiv:1504.06656.} \blue{JCAP 1511 (2015) no.11, 034}}

\vspace{0.3em}
\cvline{[11]  Large-scale clustering of Lyman-alpha emission intensity from SDSS/BOSS}{ \\ Rupert A.C. Croft \etal Cites: 8. 
		\hfill \href{http://arxiv.org/abs/1504.04088}{ArXiv:1504.04088.}  \blue{MNRAS. 457 (2016) no.4, 3541-3572}}

\vspace{0.3em}	
\cvline{[12]  A divergence-free parametrization for dynamical dark energy}{\\ Ozgur Akarsu, Tekin Dereli, JAV. 
		Cites: 5.  
		 \hfill \href{http://arxiv.org/abs/1501.07598}{ArXiv:1501.07598.} \blue{JCAP, 1506 (2015) 06, 049} }

\vspace{0.3em}	
\cvline{[13] The Eleventh and Twelfth Data Releases of the Sloan Digital Sky Survey:  Final Data from SDSS-III}
		{Shadab Alam \etal Cites: 192.  \hfill \href{http://arxiv.org/abs/1501.00963.}{ArXiv:1501.00963.} \blue{ApJs 219 (2015) 1, 12}}

\vspace{0.3em}			
\cvline{[14] Constraining Hybrid Natural Inflation with recent CMB data}
		{JAV, Mariana Carrillo, Gabriel German, Alfredo Herrera, J.C. Hidalgo. Cites: 4.
	 \hfill \href{http://arxiv.org/abs/arXiv:1411.6616}{ArXiv:1411.6616.} \blue{JCAP 1502 (2015) 02, 039}}
		
\vspace{0.3em}			
\cvline{[15] Cosmological Implications of baryon acoustic oscillation (BAO) measurements}{\\ \'Eric Aubourg \etal Cites: 118.
 	  \hfill \href{http://arxiv.org/abs/arXiv:1411.1074}{ArXiv:1411.1074.} \blue{Phys. Rev. D92 (2015) no.12, 123516}	}	

\vspace{0.3em}			
\cvline{[16] Reciprocity invariance of the Friedmann equation, missing matter and double dark energy}{JAV \etal Cites: 2. 
	\hfill \href{http://arxiv.org/abs/1208.2542}{ArXiv:1208.2542.} \blue{Submitted to PRD}} 
	
\vspace{0.3em}		
\cvline{[17] Constraints on the Tensor-to-Scalar ratio for non-power law models}{JAV, M. Bridges, Yin-Zhe Ma, M.P. Hobson. Cites: 10.
	\hfill \href{http://arxiv.org/abs/1303.4014}{ArXiv:1303.4014.}~\blue{ JCAP~08(001)~2013}} 
	
\vspace{0.3em}	
\cvline{[18] Reconstruction of the Dark Energy equation of  state}{JAV, M.P. Hobson, M. Bridges, A.N. Lasenby. Cites: 24.
	\hfill \href{http://arxiv.org/abs/1205.0847}{ArXiv:1205.0847.} \blue{JCAP, 09(020), 2012}}

\vspace{0.3em}	
\cvline{[19] Model selection applied to reconstruction of the Primordial Power Spectrum}{JAV, M.P. Hobson, M. Bridges, A.N. Lasenby. Cites: 28.
	 \hfill \href{http://arxiv.org/abs/1203.1252}{ArXiv:1203.1252.} \blue{ JCAP 006(106), 2012}}

\vspace{0.3em}	
\cvline{[20] A Bayesian study of the primordial power spectrum from a novel closed universe}{JAV, A.N. Lasenby, M.P. Hobson, M. Bridges. Cites: 12.
	\hfill \href{http://arxiv.org/abs/1103.4619}{ArXiv:1103.4619.} \blue{MNRAS 422, 1948-1956, 2011}}

\vspace{0.3em}	
\cvline{[21] Dynamics of scalar field dark matter with a cosh potential}
	{Tonatiuh Matos, Jos\'e-Rub\'en Lu\'evano, Israel Quiros, L. Arturo Urena-L\'opez,  JAV. Cites: 30.
	\hfill  \href{http://arxiv.org/abs/0906.0396}{ArXiv:0906.0396.}\blue{ PRD 80, 123521, 2009}}

\vspace{0.3em}	
\cvline{[22] Self-interacting Scalar Field Trapped in a Randall-Sundrum Braneworld}
	{Tam\'e Gonz\'alez, Tonatiuh Matos, Israel Quiros, JAV. Cites: 7.
	\hfill \href{http://arxiv.org/abs/0812.1734}{ArXiv:0812.1734.} \blue{PLB 676, 161-167, 2009}} 

\vspace{0.1em}	
\cvline{[23] $\phi^2$ as Dark Matter}{\\ Tonatiuh Matos, JAV, Juan Magana. Cites: 62.
	\hfill \href{http://arxiv.org/abs/0806.0683}{ArXiv:0806.0683.} \blue{ MNRAS 393, 1359-1369, 2008}}

\vspace{0.3em}	
\cvline{[24] An alternative Interpretation for the Moduli Fields of the Cosmology Associated to Type IIB Supergravity with Fluxes}
	{Tonatiuh Matos, Jos\'e-Rub\'en Luevano, Hugo Grac\'ia Compe\'an, \\ JAV. 
	 \hfill \href{http://arxiv.org/abs/hep-th/0511098}{ArXiv:0511098.} \blue{ IJMPA  23, 1949-1962, 2008}}

%----------------------------------------------------------------------------------------------------------------------------------------------------------
\vspace{0.0em}


\subsection{In Preparation \href{http://www.cosmo.bnl.gov/www/jvazquez/files/Research.html}{\small{\blue{(link)}}}}

\vspace{0.3em}	
	
\vspace{0.3em}
\cvline{[1p] Measurement of BAO correlations at z=2.3 with SDSS DR12 Ly$\alpha$ Forests}
	{\\ BOSS collaboration \hfill \href{https://trac.sdss3.org/browser/repo/boss/papers/LyaF/DR12Lya/pdf}{\blue{Link}}}{}

\vspace{0.3em}	
\cvline{[2p] Early Dark Energy: Reality and Fiction}{
	\\ JAV, An${ \breve{z}}$e Slosar, Hee-Jong Seo, David Weinberg.  
	\href{https://github.com/ja-vazquez/Early_DE}{\hfill \blue{Link}}} 

\vspace{0.3em}	
\cvline{[3p] Gaussian Embedding -- massively parallelizable sampling algorithm.}
	{\\ JAV, An${ \breve{z}}$e Slosar, Andreu Font-Ribera, Patrick McDonald. 
	\href{https://github.com/ja-vazquez/GM_Sampler}{\hfill \blue{Link}}}
	
\vspace{0.3em}		
\cvline{[4p] Cosmological constraints on Modified Gravity}{\\ JAV, M.P. Hobson, A.N. Lasenby, M. Bridges.
	\href{http://www.cosmo.bnl.gov/www/jvazquez/files/Research.html}{\hfill \blue{Link}}}

\vspace{0.3em}	
\cvline{[5p] Fourier-law for deceleration parameter.}{Ozgur Akarsu, Tekin Dereli, Suresh Kumar, JAV.  }
		
%----------------------------------------------------------------------------------------------------------------------------------------------------------
\vspace{0.3em}

\subsection{Conference Proceedings}
\vspace{0.3em}	

\cvline{[1C] Cosmological Implications of baryon acoustic oscillation (BAO) measurements}
	{\\ Jose Vazquez, \hfill \href{http://meetings.aps.org/Meeting/APR15/Session/Z2.5}{\blue{APS 6 No 4 (2015)}}}

\cvline{[2C] Study of Several Potentials as Scalar Field Dark Matter Candidates}
	{Tonatiuh Matos, JAV, Juan Magana. 
	\hfill AIP Conf. Proc. 1083, 144-170, 2008. \href{http://inspirehep.net/record/808386} { \blue{AIP, 808386}}}

\vspace{0.3em}	
\cvline{[3C] Alternative interpretation for the moduli fields of string theories}
	{Tonatiuh Matos, Jos\'e Rub\'en Luevano, L. Arturo Urena, JAV.  
	\hfill J. Phys. Conf. Ser. 91, 012014, 2007. \href{http://inspirehep.net/record/773227} {\blue{JP, 773227}}}
%----------------------------------------------------------------------------------------------------------------------------------------------------------
\vspace{0.3em}

\subsection{Reviews}
\cvline{[1R] Dark matter in the Universe: goals and challenges}{JAV, Tonatiuh Matos.
	Rev. Mex. de F\'isica E. 54, 193-202, 2008. 
	\hfill \href{http://www.scielo.org.mx/scielo.php?pid=S1870-35422008000200012&script=sci_abstract} {\blue{RMF, 1870-3542}}}
%	
\cvline{[2R] Constraining Cosmological Inflation}{JAV, Tonatiuh Matos.
	 \hfill \blue{Rev. Mex. Fis. E.}}
%----------------------------------------------------------------------------------------------------------------------------------------------------------
\vspace{0.3em}


\section{Invited Talks}
\cvline{02.2016}{{The current status of the Universe.}							\hfill Science Center, NY, US}	
\cvline{04.2015}{{Cosmological implications of BAO measurements: BOSS DR11.} \hfill APS, MD, US}
\cvline{}{ \textit{\small \qquad Plenary talk on behalf of the BOSS Collaboration} }
\cvline{04.2015}{{Gaussian Embedding algorithm and the BAO.} 				\hfill CMU, PA, US}
\cvline{03.2015}{{Cosmology with BAO measurements.}		 				\hfill Aspen, CO, US}
\cvline{02.2015}{{The current status of the Universe.} 						\hfill Koc University, Istanbul, TR}
\cvline{02.2015}{{The standard cosmological model: LCDM.} 					\hfill ITU, Istanbul, TR}
\cvline{01.2015}{{Gaussian Embedding algorithm and the SimpleMC code.} 			\hfill Berkeley, CA, US}
\cvline{12.2014}{{Cosmological Implications of BAO measurements.} 			\hfill SDSS Meeting, NM, US}
\cvline{}{ \textit{\small \qquad  Plenary talk on behalf of the BOSS Collaboration} }
\cvline{10.2014}{{BAO implications on Dark Energy constraints.}				\hfill  BNL, NY, US}
\cvline{08.2013}{{Model Selection applied to Dark Energy models.} 				\hfill UNAM, MX}
\cvline{09.2013}{{Dark Energy: Cosmological constant and other alternatives.} 		\hfill CINVESTAV, MX}
\cvline{04.2012}{{Comparison  of  Cosmological  Models  with  current Observations.} \hfill Cambridge, UK}{}	
%--------------------------------------------------------------------------------------------------------------------------------------------------------
\vspace{0.3em}


\subsection{Talks\small{-(past five years)}}
\cvline{10.2015}{{The current status of the Universe.}						\hfill BNL, NY, US}
\cvline{06.2014}{{BAO in the Ly-$\alpha$ forest of BOSS DR11 quasars.}			\hfill BNL, NY, US}
\cvline{09.2013}{{Dark Energy: Cosmological constant and other alternatives.}		\hfill CINVESTAV, MX}
\cvline{09.2013}{{Model Selection applied to Dark Energy models.}				\hfill UNAM, MX}
\cvline{09.2013}{{Energ\'ia oscura: alternativas a la constante cosmol\'ogica.} \hfill INAOE, Puebla, MX}{}	
\cvline{02.2013}{{Constraining alternative models with future observations.}		\hfill IF, UNAM, MX}{}	
\cvline{04.2012}{{Comparison  of  Cosmological  Models  with  current Observations.} \hfill Cambridge, UK}{}	
\cvline{01.2011}{{An overview of Statistical Cosmology.}						\hfill  ININ, MX}{}	
\cvline{01.2011}{{Constraining cosmological models with current data.}			\hfill CINVESTAV, MX}{}{}
\cvline{04.2010}{{Comparing a novel closed Universe model with  CMB data.}		\hfill KICC, Cambridge, UK}{}{}	
%\cvline{05.2008}{{Potential scalar field reconstruction.} 						\hfill CINVESTAV, MX}{}{}
%\cvline{10.2006}{{General aspects of Dark Matter.}							\hfill CINVESTAV, MX}{}{}
%\cvline{08.2005}{{Galaxy formation with Scalar Field Dark Matter.}				\hfill CCF, UNAM, MX}{}{}
%\cvline{01.2005}{{What is the Universe made of: Dark Matter.}					\hfill UAEM, MX}{}{} 

%--------------------------------------------------------------------------------------------------------------------------------------------------------

\vspace{0.3em}

\section{Hacking}
\cvline{08.2016}{PyData.													\hfill Chicago, IL, US}
\cvline{07.2016}{PyGotham.												\hfill UN, NY, US}
\cvline{07.2016}{Database Camp.											\hfill NY, US}
\cvline{06.2016}{8th Astronomical Data Analysis Summer School.					\hfill Chania,  GR}
\cvline{01.2015}{Symposium and Hack Week on data-intensive cosmology.			\hfill  Berkeley, CA, US}
\cvline{04.2015}{SciCoder 6 Workshop.										\hfill  NY, US}


\vspace{0.3em}

\section{Travel grants } 
\cvline{06.2016}{Summer School in Statistics for Astronomers.					\hfill Penn State, PA, US}
\cvline{05.2016}{Statistical Challenges in 21st Century Cosmology. 				\hfill Chania, GR}
\cvline{04.2015}{American Physical Society Meeting.							\hfill MD, US}
\cvline{08.2014}{Workshop on Cosmology from Baryons at High Redshift.			\hfill  Trieste, IT}
\cvline{08.2014}{Collaboration Meeting.		 							\hfill Cambridge, UK}
\cvline{07.2014}{SDSS-III and SDSS-IV Collaboration.						\hfill	 Salt Lake City, UT, US}
\cvline{01.2014}{Essential Cosmology for the next Generation.			 		\hfill Cabo, MX}
\cvline{10.2013}{Precision Astronomy with Fully Depleted CCDs.				\hfill BNL, NY, USA}
\cvline{08.2013}{Segunda reuni\'on de estudiantes de Astronom\'ia.				\hfill INAOE, Puebla, MX}
\cvline{07.2013}{Statistical methods applied to modern cosmology. \hfill UNAM, MX}
\cvline{05.2012}{Testing General Relativity with Astrophysical Systems.			\hfill Harvard, MA, US}
\cvline{07.2011}{New Horizons for High Redshifts.							\hfill Cambridge, UK}
\cvline{07.2011}{PASCOS 2011.										\hfill Cambridge, UK}
\cvline{01.2011}{Essential Cosmology for the Next Generation.					\hfill Jalisco, MX}
\cvline{12.2010}{Fourth TRR33 Winter School.								\hfill  Passo del Tonale, IT}
\cvline{07.2008}{Summer school in Cosmology.								\hfill  ICTP, Trieste, IT}
\cvline{05.2008}{III International Meeting on Gravitation and Cosmology.			\hfill Morelia, MX}{}
\cvline{09.2007}{Latin-American School of Physics.							\hfill DF, MX}{}
\cvline{08.2007}{XXXV SLAC Summer Institute.							\hfill Stanford, CA, USA}{}
\cvline{06.2007}{International Conference on Quantum Gravity.					\hfill Morelia, MX}{}
\cvline{07.2006}{New Frontiers in Numerical Relativity.						\hfill AIE, Berlin, DE}{}
\cvline{07.2004}{XIII Summer at the National Astronomic Observatory.			\hfill Ensenada, MX}{}
%-----------------------------------------------------------------------------------------------------------------------------------------------------------------
\vspace{0.3em}

\subsection{Domestic} 
\cvline{08.2009}{Cluster de Alto desempeno.								\hfill UAEH, Hidalgo, MX}{}{}
\cvline{02.2008}{1er Congreso de Cosmolog\'ia.							\hfill IFUG, MX}{}{}{}
\cvline{09.2007}{2a Reuni\'on del Instituto Avanzado en Cosmolog\'ia.			\hfill CRyA-UNAM, MX}{}{}{}
\cvline{07.2007}{Advanced Summer School in Physics.						\hfill CINVESTAV, MX}{}{}{}
\cvline{04.2007}{XV Reuni\'on anual de la divisi\'on de Gravitaci\'on y F\'isica Matem\'atica.\hfill IPN, MX}{}{}{}
\cvline{01.2007}{Obreg\'on Fest.										\hfill IFUG, MX}{}{}{}	
\cvline{01.2007}{1era Reuni\'on  Instituto Avanzado de Cosmolog\'ia.				\hfill UNAM, MX}{}{}{}		
\cvline{11.2006}{VII Mexican School on Gravitation.							\hfill Playa del Carmen, MX}{}
\cvline{04.2006}{XIV Reuni\'on Anual de la Divisi\'on de Gravitaci\'on y F\'isica.		\hfill CINVESTAV, MX}{}{}{}
\cvline{07.2005}{ IV Mexican School of Astrophysics, EMA 05.					\hfill Morelia, MX}{}{}{}
\cvline{09.2003}{3rd. Workshop Optica Moderna.							\hfill INAOE, Puebla, MX}{}{}{}
\cvline{08.2003}{XI Summer School on Physics, La visi\'on molecular de la materia.	\hfill UAEM, Morelos, MX}{}{}{}		
\cvline{08.2002}{X Summer School on Physics, La visi\'on molecular de la materia.	\hfill UAEM, Morelos, MX}{}{}{}	
%------------------------------------------------------------------------------------------------------------


\vspace{0.3em}


 \section{Organization }
\cvline{09.2013}{Workshop Organiser: Statistical and Numerical methods  in Cosmology. \hfill IF, UNAM, MX}{}{}{} 
\cvline{01.2011}{Mini-workshop Organiser: overview to CAMB and CosmoMC. \hfill  ININ, MX}{}{}{}
\cvline{2007-2008}{Seminar Organiser, ``Geometry and Gravitation''. \hfill CINVESTAV, MX}{}{}{}
\cvline{2005-2007}{Seminar  Organiser, ``Cosmology, Astrophysics and Numerical R''. \hfill CINVESTAV, MX}{}{}{}	
\cvline{2004-2005}{Committee Member, ``Consejo T\'ecnico''. \hfill UAEM, MX}{}{}{}
\cvline{2004-2005}{Committee Member, ``Consejo Estudiantil de la Sociedad de Alumnos''. \hfill UAEM, MX}{}{}{}
\cvline{2001-2002}{Committee Member, ``Consejo Estudiantil de la Sociedad de Alumnos''. \hfill UAEM, MX}{}{}{}
%------------------------------------------------------------------------------------------------------------
\vspace{0.3em}

\subsection{Teaching and Outreach}
 \cvline{08.2016}{Mentoring  a summer high school student, BNL.}
 \cvline{10.2015}{Mentoring  a summer high school student, BNL.}
 \cvline{07.2013}{Tutor of three Master summer students, CINVESTAV.}
 \cvline{2006 }{Graduate Research Assistant, \emph{Photo Acoustic Spectroscopy}, CINVESTAV.}
 \cvline{2004-2005}{Undergraduate Research Assistantship,\emph{Galaxy Formation with dark matter}, UAEM.}{}{}{}
 \cvline{2004 }{Undergraduate Teaching Assistant, \emph{Mechanics Subject}, UAEM.}
 \cvline{2003-2004}{Undergraduate Research Assistantship,\emph{\small Opto-galvatinic spectroscopy of plasmas to low temperature},  {\small UAEM.}} 	
 %\cvline{2001 {\tiny *}}{Undergraduate Teaching Assistant, \emph{Calculus Subject}, High School.}{\hspace{2cm}* \small Six month period.}{}	
%------------------------------------------------------------------------------------------------------------

%----------------------------------------------------------------------------------------------------------------------------------------------------------
\vspace{0.3em}


\section{Skills and Interests}
\cvline{Programming Languages}{ \hfill Python, C/C++, Fortran, R, Bash Scripting}{}
\cvline{Maths}{\hfill Maple, Mathematica, Matlab (basic)}{}
\cvline{Op. Systems}{\hfill Linux, Windows, Mac OS X}
\cvline{Design}{\hfill Latex, HTML, CSS}
\cvline{Databases}{\hfill MySQL, SQLite}
\cvline{Useful}{\hfill Gnuplot, Git, SVN}

\vspace{0.3em}
\subsection{Packages, libraries and frameworks}
\cvline{Python}{\hfill Numpy, Pandas, Scipy, Scikit-learn, Beatiful Soup, Matplotlib, Bokeh, Seaborn, Flask.}
\cvline{R}{\hfill dplyr, Main ones for Stats and ML, ggplot2, Shiny}
\cvline{C/C++, Fortran}{\hfill  LAPACK, OpenMP, MPI}
\cvline{HPC Clusters}{\hfill NERSC(LBNL), Astro (BNL), Darwin (Cambridge), LaSuma-(CINVESTAV)}


\vspace{0.3em}

\subsection{Cosmology}
\cvline{}{ CAMB, CosmoMC, MultiNest, CosmoNet, CosmoSIS, SimpleMC.}
\vspace{0.3em}
\cvline{Contributions}{ 
\hfill MCMC for BAO analysis for the BOSS collaboration (Python) - \href{https://github.com/ja-vazquez/SimpleMC}{\blue{SimpleMC}}\\. 
\hfill Massively parallelizable Gaussian Embedding Sampling (Python) - \href{https://github.com/ja-vazquez/GM_Sampler}{\blue{GM algorithm}}\\.
\hfill Model Independent Bayesian Reconstruction (Fortran) - \href{http://www.mrao.cam.ac.uk/facilities/software/np-camb/}{\blue{NP-CAMB}}}
\hfill Lyman-$\alpha$ analysis for the BOSS collaboration (C++) - \href{http://c3.lbl.gov:8000/Trac.Cosmology/wiki/CosmologyCodeInstallation}{\blue{Cosmology}}

\vspace{0.3em}
%------------------------------------------------------------------------------------------------------------

\subsection{Non-Academic Projects}
\cvline{}{Scraping the web, Using APIs, Data manipulation with Pandas and SQL, 
	Playing with Stats and ML algorithms and Visualizations.}
\cvline{Meetups}{I regularly attend NYC meetups with keywords such as Python, R, SQL, Data science. }

\vspace{0.3em}
 \cvline{}{For further details see: 
	\href{https://github.com/ja-vazquez}{\blue {GitHub}},
	\href{https://bitbucket.org/ja_vazquez/}{\blue {Bitbucket}}.
	}

\vspace{0.3em}

\subsection{Others}
\vspace{0.3em}
\cvline{Languages}{\hfill  Spanish (Native);  English (Fluent);  German (Elementary). }{}
\cvline{ Sports}{\hfill Football (participation on national tournaments), Squash, Climbing, Jogging, Cycling.}
\cvline{}{\hfill Organiser of the national football tournament of Mexican Societies  in UK (05.2010).}
\cvline{Others}{ \hfill Reading: Economy,  Science, Science Fiction; Board games: Chess, Backgammon, Poker.}{}	
%\cvline{Extra}{Online Stocks/ETFs trader.}

\closesection{}                   % needed to renewcommands
\renewcommand{\listitemsymbol}{-} % change the symbol for lists
%------------------------------------------------------------------------------------------------------------	
	

\vspace{0.5em}
\section{References}
\cventry{anze@bnl.gov}{Upton, 11973, NY, US. Tel: +1 (631) 344 8012.}
{An${ \breve{z}}$e Slosar}{Brookhaven National Lab}{}{}{}
	
\vspace{0.5em}
\cventry{mph@mrao.cam.ac.uk}{Cavendish Laboratory, CB3 0HE, UK. Tel: +44 1223 339992.}
{Mike Hobson}{University of Cambridge}{}{} 

\vspace{0.5em}
\cventry{a.n.lasenby@mrao.cam.ac.uk}{Kavli Institute for Cosmology, CB3 0HA, UK. Tel: +44 1223 337293.}
{Anthony Lasenby}{University of Cambridge}{}{} 

\vspace{0.5em}
\cventry{tmatos@fis.cinvestav.mx}{Mexico D.F, 14-740 07000, MX. Tel: +52 55 5747 3834.}
{Tonatiuh Matos}{CINVESTAV}{}{}{}{}


\vspace{4.em}
\cvline{}{\hfill JAV}{}{}{}
%------------------------------------------------------------------------------------------------------------
%------------------------------------------------------------------------------------------------------------
%------------------------------------------------------------------------------------------------------------
%------------------------------------------------------------------------------------------------------------	
%\section{Current Research}
%{The research I am currently carrying out has two main  aims: The first is to generate
%predictions of  observable quantities, within cosmological models incorporating extra
%features beyond the standard Lambda Cold Dark Matter ($\Lambda$CDM). The predictions
%are sought for the  expansion history  and the generation and evolution of perturbations.
%Secondly, advanced  Bayesian  techniques, involving the calculation of evidence and 
%nested sampling,  are carried out to  compare cosmological models using  actual data, including
%the Cosmic Microwave Background, Large-Scale  Structure and Supernovae. Future  surveys
%like  Planck and CMBPol  are also  taking  part in the analysis.}
%\vspace{0.5em}
	
%------------------------------------------------------------------------------------------------------------		
	
%\section{Personal Information}
%\cvline{Name}{Jos\'e-Alberto V\'azquez-Gonz\'alez}
%\cvline{Date of birth}{September 6, 1982} 
%\cvline{Place }{Cuernavaca, Morelos, Mexico}
%\cvline{Interests}{ BAO, CMB, Ly-$\alpha$ forest; Dark Energy, Inflation; Data analysis.} 
%\vspace{0.5em}
\end{document}

